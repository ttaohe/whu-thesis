% \documentclass[type = bachelor]{whu-thesis}
\documentclass[type = master,class = academic]{whu-thesis}
% \documentclass[type = doctor]{whu-thesis}
% type: 学位类型,可选项为 bachelor, master, doctor
% class: 学位类别,可选项为 academic, professional
% showframe: 显示页面布局框架

% 以下仅列举了部分可能用到的设置选项,更多用法请参考文档《whu-thesis.pdf》

% \PassOptionsToPackage{gbnamefmt = lowercase}{biblatex} % 英文作者姓名不强制大写

\whusetup{
  info = {
    title      = {非完备观测下无人机多视角时空协同的目标检测方法研究}, % 标题,可使用 \\ 手动换行
    title*     = {Research on Object Detection Method of UAV Multi-view Spatio-temporal Collaboration under Incomplete Observation}, % 英文标题
    department  = {测绘遥感信息工程全国重点实验室},
    department* = {\parbox{0.9\linewidth}{\centering
    \linespread{0.7}\selectfont
    State Key Laboratory of Information Engineering\\
    in Surveying, Mapping and Remote Sensing\\[4pt]}},
    author     = {何涛},
    author*    = {Tao He},
    student-id = {2023206190029},
    supervisor  = {孙开敏},
    supervisor* = {Kaimin Sun},
    academic-title  = {教授},
    academic-title* = {Prof},
    % supervisor-outer = {王某某}, % 校外导师(非必填)
    % academic-title-outer = {高级工程师}, % 校外导师职称(非必填)
    subject = {计算机科学与技术}, % 学科名称(非必填)
    major   = {计算机应用技术},
    major*  = {Computer Application Technology},
    research-area  = {目标检测},
    research-area* = {Object Detection},
    year = 2026,
    month = 4,
    % clc = , % 分类号
    % udc = ,
    keywords  = {\LaTeX{}, 毕业论文, 模版, 武汉大学}, % 中文关键词
    keywords* = {\LaTeX{}, Thesis, Template, Wuhan~University}, % 英文关键词
  },
  style = {
    % 字体相关选项
    font = termes, % 西文字体,可选项为 default, times, xits, termes
    math-font = termes, % 数学字体,可选项为 default, xits, termes
    cjk-font = windows, % 中文字体,可选项为 windows, mac, fandol(Linux/Overleaf/TexPage), sourcehan, none
    % cjk-fakefont = true, % 使用伪粗体与伪斜体
    % 参考文献及引用相关选项
    bib-backend = bibtex, % 参考文献引擎,可选项为 bibtex, biblatex
    bib-style = numerical, % 参考文献样式,可选项为 numerical, author-year
    cite-style = super,
    % cite-style = <>, % 引用样式(自定义)
    bib-resource = {ref/bachelor-refs.bib}, % 参考文献数据源
    % 页面相关选项
    % chapter-page-header = true, % 章节首页是否有页眉
    % bachelor-encover = true, % 本科毕业论文英文封面
    % library, % 图书馆模式(去掉论文中所有的空白页)
    % license, % 使用授权协议书
    % fullwidth-stop = true, % 句号样式
    % footnote-style = <>, % 脚注编号样式
    % abstract-keywords-type  = blankline, % 摘要与关键词之间样式,可选项为 blankline, newline, vfill
    % abstract-keywords-type* = blankline, % 摘要与关键词之间样式,可选项为 blankline, newline, vfill
  }
}
\hypersetup{hidelinks}
\whumodule{algorithm2e}
\begin{document}
% \raggedbottom % 使得空白都置于每一页底部,可参考 https://github.com/whutug/whu-thesis/issues/276

\tableofcontents % 目录
% \listoffigures % 图目录
% \listoftables % 表目录

% 符号表
% \begin{notation}
%   $\omega_n$ & $n$-维欧氏空间中单位球的表面积 \\
%   $\alpha_n$ & $n$-维欧氏空间中单位球的体积 \\
% \end{notation}

\mainmatter

\iffalse
\chapter{第一章 绪论}
\fi
\whuchapter{1}{绪论}

\section{研究背景与意义}


\section{国内外研究现状及趋势}

\subsection{单视角时序目标检测}

\subsection{多视角空间目标检测}

\subsection{多视角时空联合目标检测}

\section{主要研究内容}

\section{论文章节安排}

测试 \cite{whu-bachelor:1}

\iffalse
\chapter{第二章 相关理论基础与难点分析}
\fi
\whuchapter{2}{相关理论基础与难点分析}

\section{深度学习基础理论}

\section{单视角时序目标检测理论}

\section{多视角时空协同目标检测难点分析}

\section{本章小结}

测试 \cite{whu-bachelor:1}

\include{pages/chapter3.tex}
\iffalse
\chapter{第四章 联合对齐的多视角空间特征补全}
\fi
\whuchapter{4}{联合对齐的多视角空间特征补全}

\section{问题定义与难点分析}

\section{网络结构设计}

\section{训练策略与实现细节}

\section{实验与结果分析}
\subsection{对比实验}
\subsection{消融实验}
\subsection{可视化分析}

\section{本章小结}

测试 \cite{whu-bachelor:1}

\iffalse
\chapter{第五章 基于时空注意力的多视角协同检测方法}
\fi
\whuchapter{5}{基于时空注意力的多视角协同检测方法}

\section{问题定义与难点分析}

\section{网络结构设计}

\section{训练策略与实现细节}

\section{实验与结果分析}
\subsection{对比实验}
\subsection{消融实验}
\subsection{可视化分析}

\section{本章小结}

测试 \cite{whu-bachelor:1}

\iffalse
\chapter{第六章 总结与展望}
\fi
\whuchapter{6}{总结与展望}

\section{总结}

\section{展望}

% 当然你也可以直接在这里写,不过这样不太方便管理
% \chapter{BBBB}


% 参考文献
% \nocite{*}
\printbibliography

% 发表的与学位论文相关的科研成果目录
% \include{pages/achievement.tex}

% 致谢
\begin{acknowledgements}
  致谢是以简短的文字对课题研究与论文撰写过程中直接给予帮助的人员(例如指导教师、答疑教师及其他人员)表示谢意。致谢是对他人劳动的尊重,也是学术规范。内容限一页。
\end{acknowledgements}

% 附录
\appendix

\include{pages/appendix.tex}

\end{document}
